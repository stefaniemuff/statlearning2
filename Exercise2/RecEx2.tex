% Options for packages loaded elsewhere
\PassOptionsToPackage{unicode}{hyperref}
\PassOptionsToPackage{hyphens}{url}
\PassOptionsToPackage{dvipsnames,svgnames,x11names}{xcolor}
%
\documentclass[
]{article}
\usepackage{amsmath,amssymb}
\usepackage{iftex}
\ifPDFTeX
  \usepackage[T1]{fontenc}
  \usepackage[utf8]{inputenc}
  \usepackage{textcomp} % provide euro and other symbols
\else % if luatex or xetex
  \usepackage{unicode-math} % this also loads fontspec
  \defaultfontfeatures{Scale=MatchLowercase}
  \defaultfontfeatures[\rmfamily]{Ligatures=TeX,Scale=1}
\fi
\usepackage{lmodern}
\ifPDFTeX\else
  % xetex/luatex font selection
\fi
% Use upquote if available, for straight quotes in verbatim environments
\IfFileExists{upquote.sty}{\usepackage{upquote}}{}
\IfFileExists{microtype.sty}{% use microtype if available
  \usepackage[]{microtype}
  \UseMicrotypeSet[protrusion]{basicmath} % disable protrusion for tt fonts
}{}
\makeatletter
\@ifundefined{KOMAClassName}{% if non-KOMA class
  \IfFileExists{parskip.sty}{%
    \usepackage{parskip}
  }{% else
    \setlength{\parindent}{0pt}
    \setlength{\parskip}{6pt plus 2pt minus 1pt}}
}{% if KOMA class
  \KOMAoptions{parskip=half}}
\makeatother
\usepackage{xcolor}
\usepackage[margin=1in]{geometry}
\usepackage{color}
\usepackage{fancyvrb}
\newcommand{\VerbBar}{|}
\newcommand{\VERB}{\Verb[commandchars=\\\{\}]}
\DefineVerbatimEnvironment{Highlighting}{Verbatim}{commandchars=\\\{\}}
% Add ',fontsize=\small' for more characters per line
\usepackage{framed}
\definecolor{shadecolor}{RGB}{248,248,248}
\newenvironment{Shaded}{\begin{snugshade}}{\end{snugshade}}
\newcommand{\AlertTok}[1]{\textcolor[rgb]{0.94,0.16,0.16}{#1}}
\newcommand{\AnnotationTok}[1]{\textcolor[rgb]{0.56,0.35,0.01}{\textbf{\textit{#1}}}}
\newcommand{\AttributeTok}[1]{\textcolor[rgb]{0.13,0.29,0.53}{#1}}
\newcommand{\BaseNTok}[1]{\textcolor[rgb]{0.00,0.00,0.81}{#1}}
\newcommand{\BuiltInTok}[1]{#1}
\newcommand{\CharTok}[1]{\textcolor[rgb]{0.31,0.60,0.02}{#1}}
\newcommand{\CommentTok}[1]{\textcolor[rgb]{0.56,0.35,0.01}{\textit{#1}}}
\newcommand{\CommentVarTok}[1]{\textcolor[rgb]{0.56,0.35,0.01}{\textbf{\textit{#1}}}}
\newcommand{\ConstantTok}[1]{\textcolor[rgb]{0.56,0.35,0.01}{#1}}
\newcommand{\ControlFlowTok}[1]{\textcolor[rgb]{0.13,0.29,0.53}{\textbf{#1}}}
\newcommand{\DataTypeTok}[1]{\textcolor[rgb]{0.13,0.29,0.53}{#1}}
\newcommand{\DecValTok}[1]{\textcolor[rgb]{0.00,0.00,0.81}{#1}}
\newcommand{\DocumentationTok}[1]{\textcolor[rgb]{0.56,0.35,0.01}{\textbf{\textit{#1}}}}
\newcommand{\ErrorTok}[1]{\textcolor[rgb]{0.64,0.00,0.00}{\textbf{#1}}}
\newcommand{\ExtensionTok}[1]{#1}
\newcommand{\FloatTok}[1]{\textcolor[rgb]{0.00,0.00,0.81}{#1}}
\newcommand{\FunctionTok}[1]{\textcolor[rgb]{0.13,0.29,0.53}{\textbf{#1}}}
\newcommand{\ImportTok}[1]{#1}
\newcommand{\InformationTok}[1]{\textcolor[rgb]{0.56,0.35,0.01}{\textbf{\textit{#1}}}}
\newcommand{\KeywordTok}[1]{\textcolor[rgb]{0.13,0.29,0.53}{\textbf{#1}}}
\newcommand{\NormalTok}[1]{#1}
\newcommand{\OperatorTok}[1]{\textcolor[rgb]{0.81,0.36,0.00}{\textbf{#1}}}
\newcommand{\OtherTok}[1]{\textcolor[rgb]{0.56,0.35,0.01}{#1}}
\newcommand{\PreprocessorTok}[1]{\textcolor[rgb]{0.56,0.35,0.01}{\textit{#1}}}
\newcommand{\RegionMarkerTok}[1]{#1}
\newcommand{\SpecialCharTok}[1]{\textcolor[rgb]{0.81,0.36,0.00}{\textbf{#1}}}
\newcommand{\SpecialStringTok}[1]{\textcolor[rgb]{0.31,0.60,0.02}{#1}}
\newcommand{\StringTok}[1]{\textcolor[rgb]{0.31,0.60,0.02}{#1}}
\newcommand{\VariableTok}[1]{\textcolor[rgb]{0.00,0.00,0.00}{#1}}
\newcommand{\VerbatimStringTok}[1]{\textcolor[rgb]{0.31,0.60,0.02}{#1}}
\newcommand{\WarningTok}[1]{\textcolor[rgb]{0.56,0.35,0.01}{\textbf{\textit{#1}}}}
\usepackage{graphicx}
\makeatletter
\def\maxwidth{\ifdim\Gin@nat@width>\linewidth\linewidth\else\Gin@nat@width\fi}
\def\maxheight{\ifdim\Gin@nat@height>\textheight\textheight\else\Gin@nat@height\fi}
\makeatother
% Scale images if necessary, so that they will not overflow the page
% margins by default, and it is still possible to overwrite the defaults
% using explicit options in \includegraphics[width, height, ...]{}
\setkeys{Gin}{width=\maxwidth,height=\maxheight,keepaspectratio}
% Set default figure placement to htbp
\makeatletter
\def\fps@figure{htbp}
\makeatother
\setlength{\emergencystretch}{3em} % prevent overfull lines
\providecommand{\tightlist}{%
  \setlength{\itemsep}{0pt}\setlength{\parskip}{0pt}}
\setcounter{secnumdepth}{-\maxdimen} % remove section numbering
\ifLuaTeX
  \usepackage{selnolig}  % disable illegal ligatures
\fi
\usepackage{bookmark}
\IfFileExists{xurl.sty}{\usepackage{xurl}}{} % add URL line breaks if available
\urlstyle{same}
\hypersetup{
  pdftitle={Module 2: Recommended Exercises},
  pdfauthor={Sara Martino, Stefanie Muff, Kenneth Aase; Department of Mathematical Sciences, NTNU},
  colorlinks=true,
  linkcolor={Maroon},
  filecolor={Maroon},
  citecolor={Blue},
  urlcolor={blue},
  pdfcreator={LaTeX via pandoc}}

\title{Module 2: Recommended Exercises}
\usepackage{etoolbox}
\makeatletter
\providecommand{\subtitle}[1]{% add subtitle to \maketitle
  \apptocmd{\@title}{\par {\large #1 \par}}{}{}
}
\makeatother
\subtitle{TMA4268 Statistical Learning V2025}
\author{Sara Martino, Stefanie Muff, Kenneth Aase \and Department of
Mathematical Sciences, NTNU}
\date{January 22, 2025}

\begin{document}
\maketitle

\subsection{Problem 1}\label{problem-1}

\begin{enumerate}
\def\labelenumi{\alph{enumi})}
\tightlist
\item
  Describe a real-life application in which \emph{classification} might
  be useful. Identify the response and the predictors. Is the goal
  inference or prediction?
\item
  Describe a real-life application in which \emph{regression} might be
  useful. Identify the response and the predictors. Is the goal
  inference or prediction?
\end{enumerate}

\subsection{Problem 2}\label{problem-2}

\begin{enumerate}
\def\labelenumi{\alph{enumi})}
\tightlist
\item
  Take a look at Figure 2.9 in the course book (p.~31). Do the flexible
  or rigid methods have the highest test error? Is this always the case?
\item
  Does a very small variance imply that the data has been under- or
  overfit?
\item
  Relate the problem of over- and underfitting to the bias-variance
  trade-off.
\end{enumerate}

\subsection{Problem 3 -- Exercise 2.4.9 from ISL textbook
(modified)}\label{problem-3-exercise-2.4.9-from-isl-textbook-modified}

This exercise involves the \texttt{Auto} dataset from the \texttt{ISLR}
library. Load the data into your R session by running the following
commands:

\begin{Shaded}
\begin{Highlighting}[]
\FunctionTok{library}\NormalTok{(ISLR)}
\FunctionTok{data}\NormalTok{(Auto)}
\end{Highlighting}
\end{Shaded}

PS: if the \texttt{ISLR} package is not installed (\texttt{library}
function gives error) you can install it by running
\texttt{install.packages("ISLR")} before you load the package the first
time.

\begin{enumerate}
\def\labelenumi{\alph{enumi})}
\item
  View the data. What are the dimensions of the data? Which predictors
  are quantitative and which are qualitative?
\item
  What is the range (min, max) of each quantitative predictor? Hint: use
  the \texttt{range()} function. For more advanced users, check out
  \texttt{sapply()}.
\item
  What is the sample mean and sample standard deviation for each
  quantitative predictor?
\item
  Now, make a new dataset called \texttt{ReducedAuto} where you remove
  the 10th through 85th observations. What is the range, mean and
  standard deviation of the quantitative predictors in this reduced set?
\item
  Using the full dataset, investigate the quantitative predictors
  graphically using a scatter plot. Do you see any strong relationships
  between the predictors? Hint: try out the \texttt{ggpairs()} function
  from the \texttt{GGally} package.
\item
  Suppose we wish to predict gas mileage (\texttt{mpg}) on the basis of
  the other variables (both quantitative and qualitative). Make some
  plots showing the relationships between \texttt{mpg} and the
  qualitative predictors (hint: \texttt{geom\_boxplot()}). Which
  predictors would you consider helpful when predicting \texttt{mpg}?
\item
  The correlation of two variables \(X\) and \(Y\) are defined as \[
  \text{cor}(X,Y) = \frac{\text{cov}(X,Y)}{\sigma_X\sigma_Y}.
  \] The correlation matrix and covariance matrix can be easily found in
  \texttt{R} with the \texttt{cor()} and \texttt{cov()} functions,
  respectively. Use only the covariance matrix (as shown below) to find
  the correlation between \texttt{mpg} and \texttt{displacement},
  \texttt{mpg} and \texttt{horsepower}, and \texttt{mpg} and
  \texttt{weight}. Do your results coincide with the correlation matrix
  you find using \texttt{cor(Auto{[},\ quant{]})}?
\end{enumerate}

\begin{Shaded}
\begin{Highlighting}[]
\NormalTok{quant }\OtherTok{\textless{}{-}} \DecValTok{1}\SpecialCharTok{:}\DecValTok{7}
\NormalTok{covMat }\OtherTok{\textless{}{-}} \FunctionTok{cov}\NormalTok{(Auto[, quant])}
\end{Highlighting}
\end{Shaded}

\subsection{Problem 4 -- Multivariate normal
distribution}\label{problem-4-multivariate-normal-distribution}

The pdf of a multivariate normal distribution is on the form
\[ f(\boldsymbol{x}) = \frac{1}{(2\pi)^{p/2}|\boldsymbol{\Sigma|}} \exp\{-\frac{1}{2}(\boldsymbol{x-\mu})^T\boldsymbol{\Sigma}^{-1}(\boldsymbol{x-\mu)}\},\]
where \(\bf{x}\) is a random vector of size \(p\times 1\),
\(\boldsymbol{\mu}\) is the mean vector of size \(p\times 1\) and
\(\boldsymbol{\Sigma}\) is the covariance matrix of size \(p\times p\).

\begin{enumerate}
\def\labelenumi{\alph{enumi})}
\item
  Use the \texttt{mvrnorm()} function from the \texttt{MASS} library to
  simulate 1000 values from multivariate normal distributions with
\item
  \[ \boldsymbol{\mu} = \begin{pmatrix}
  2 \\
  3 
  \end{pmatrix} \quad \text{and} \quad \boldsymbol{\Sigma} = \begin{pmatrix}
  1 & 0\\
  0 & 1
  \end{pmatrix},\]
\end{enumerate}

\begin{enumerate}
\def\labelenumi{\roman{enumi})}
\setcounter{enumi}{1}
\item
  \[ \boldsymbol{\mu} = \begin{pmatrix}
  2 \\
  3 
  \end{pmatrix} \quad \text{and} \quad \boldsymbol{\Sigma} = \begin{pmatrix}
  1 & 0\\
  0 & 5
  \end{pmatrix},\]
\item
  \[ \boldsymbol{\mu} = \begin{pmatrix}
  2 \\
  3 
  \end{pmatrix} \quad \text{and} \quad \boldsymbol{\Sigma} = \begin{pmatrix}
  1 & 2\\
  2 & 5
  \end{pmatrix},\]
\item
  \[ \boldsymbol{\mu} = \begin{pmatrix}
  2 \\
  3 
  \end{pmatrix} \quad \text{and} \quad \boldsymbol{\Sigma} = \begin{pmatrix}
  1 & -2\\
  -2 & 5
  \end{pmatrix}.\]
\end{enumerate}

\begin{enumerate}
\def\labelenumi{\alph{enumi})}
\setcounter{enumi}{1}
\tightlist
\item
  Make a scatter plot of the four sets of simulated data sets. Can you
  see which plot belongs to which distribution?
\end{enumerate}

\subsection{Problem 5 -- Theory and practice: training and test MSE;
bias-variance}\label{problem-5-theory-and-practice-training-and-test-mse-bias-variance}

We will now look closely into the simulations and calculations performed
for the training error (\texttt{trainMSE}), test error
(\texttt{testMSE}), and the bias-variance trade-off in lecture 1 of
module 2.

Below, the code to run the simulation is included. The data is simulated
according to the following specifications:

\begin{itemize}
\tightlist
\item
  True function \(f(x)=x^2\) with normal noise
  \(\varepsilon \sim N(0,2^2)\).
\item
  \(x= -2.0, -1.9, ... ,4.0\) (grid with 61 values).
\item
  Parametric models are fitted (polynomials of degree 1 to degree 20).
\item
  M=100 simulations.
\end{itemize}

\subsubsection{a) Problem set-up}\label{a-problem-set-up}

Look at the code below, copy it and run it yourself. Explain roughly
what is done (you do not need not understand the code in detail), for
example by commenting the code after copying it into your report.

We will learn more about the \texttt{lm} function in Module 3 - now just
think of this as fitting a polynomial regression and then predict gives
the fitted curve in our grid points. \texttt{predictions\_list} is just
a way to save \(M\) simulations of 61 grid-points in \(x\) and 20
polynomial models.

\begin{Shaded}
\begin{Highlighting}[]
\FunctionTok{set.seed}\NormalTok{(}\DecValTok{2}\NormalTok{) }\CommentTok{\# to reproduce}

\NormalTok{M }\OtherTok{\textless{}{-}} \DecValTok{100} \CommentTok{\# repeated samplings, x fixed}
\NormalTok{nord }\OtherTok{\textless{}{-}} \DecValTok{20} \CommentTok{\# order of polynomials}

\CommentTok{\#{-}{-}{-}{-}{-}{-}}

\NormalTok{x }\OtherTok{\textless{}{-}} \FunctionTok{seq}\NormalTok{(}\AttributeTok{from =} \SpecialCharTok{{-}}\DecValTok{2}\NormalTok{, }\AttributeTok{to =} \DecValTok{4}\NormalTok{, }\AttributeTok{by =} \FloatTok{0.1}\NormalTok{)}

\NormalTok{truefunc }\OtherTok{\textless{}{-}} \ControlFlowTok{function}\NormalTok{(x) \{}
  \FunctionTok{return}\NormalTok{(x }\SpecialCharTok{\^{}} \DecValTok{2}\NormalTok{)}
\NormalTok{\}}
\NormalTok{true\_y }\OtherTok{\textless{}{-}} \FunctionTok{truefunc}\NormalTok{(x)}
\NormalTok{error }\OtherTok{\textless{}{-}} \FunctionTok{matrix}\NormalTok{(}\FunctionTok{rnorm}\NormalTok{(}\FunctionTok{length}\NormalTok{(x) }\SpecialCharTok{*}\NormalTok{ M, }\AttributeTok{mean =} \DecValTok{0}\NormalTok{, }\AttributeTok{sd =} \DecValTok{2}\NormalTok{),}
                \AttributeTok{nrow =}\NormalTok{ M,}
                \AttributeTok{byrow =} \ConstantTok{TRUE}\NormalTok{)}
\NormalTok{ymat }\OtherTok{\textless{}{-}} \FunctionTok{matrix}\NormalTok{(}\FunctionTok{rep}\NormalTok{(true\_y, M), }\AttributeTok{byrow =} \ConstantTok{TRUE}\NormalTok{, }\AttributeTok{nrow =}\NormalTok{ M) }\SpecialCharTok{+}\NormalTok{ error  }\CommentTok{\# Each row is a simulation}

\CommentTok{\#{-}{-}{-}{-}{-}{-}}

\NormalTok{predictions\_list }\OtherTok{\textless{}{-}} \FunctionTok{lapply}\NormalTok{(}\DecValTok{1}\SpecialCharTok{:}\NormalTok{nord, matrix, }\AttributeTok{data =} \ConstantTok{NA}\NormalTok{, }\AttributeTok{nrow =}\NormalTok{ M, }\AttributeTok{ncol =} \FunctionTok{ncol}\NormalTok{(ymat))}
\ControlFlowTok{for}\NormalTok{ (i }\ControlFlowTok{in} \DecValTok{1}\SpecialCharTok{:}\NormalTok{nord) \{}
  \ControlFlowTok{for}\NormalTok{ (j }\ControlFlowTok{in} \DecValTok{1}\SpecialCharTok{:}\NormalTok{M) \{}
\NormalTok{    predictions\_list[[i]][j, ] }\OtherTok{\textless{}{-}} \FunctionTok{predict}\NormalTok{(}\FunctionTok{lm}\NormalTok{(ymat[j, ] }\SpecialCharTok{\textasciitilde{}} \FunctionTok{poly}\NormalTok{(x, i, }\AttributeTok{raw =} \ConstantTok{TRUE}\NormalTok{)))}
\NormalTok{  \}}
\NormalTok{\}}

\CommentTok{\# Plotting {-}{-}{-}{-}{-}}

\FunctionTok{library}\NormalTok{(tidyverse) }\CommentTok{\# The tidyverse contains ggplot2, as well as tidyr and dplyr, }
\CommentTok{\# which we can use for dataframe manipulation.}

\NormalTok{list\_of\_matrices\_with\_deg\_id }\OtherTok{\textless{}{-}} 
  \FunctionTok{lapply}\NormalTok{(}\DecValTok{1}\SpecialCharTok{:}\NormalTok{nord, }
         \ControlFlowTok{function}\NormalTok{(poly\_degree) \{}
           \FunctionTok{cbind}\NormalTok{(predictions\_list[[poly\_degree]],}
                 \AttributeTok{simulation\_num =} \DecValTok{1}\SpecialCharTok{:}\NormalTok{M,}
\NormalTok{                 poly\_degree)}
\NormalTok{         \}}
\NormalTok{  )}
\CommentTok{\# Now predictions\_list is a list with 20 entries, where each entry is a matrix }
\CommentTok{\# with 100 rows, where each row is the predicted polynomial of that degree.}
\CommentTok{\# We also have a column for the simulation number, and a column for polynomial degree.}

\CommentTok{\# Extract each matrix and bind them to one large matrix}
\NormalTok{stacked\_matrices }\OtherTok{\textless{}{-}}  \ConstantTok{NULL}
\ControlFlowTok{for}\NormalTok{ (i }\ControlFlowTok{in} \DecValTok{1}\SpecialCharTok{:}\NormalTok{nord) \{}
\NormalTok{  stacked\_matrices }\OtherTok{\textless{}{-}}
    \FunctionTok{rbind}\NormalTok{(stacked\_matrices, list\_of\_matrices\_with\_deg\_id[[i]])}
\NormalTok{\}}
\NormalTok{stacked\_matrices\_df }\OtherTok{\textless{}{-}} \FunctionTok{as.data.frame}\NormalTok{(stacked\_matrices)}

\CommentTok{\# Convert from wide to long (because that is the best format for ggplot2)}
\NormalTok{long\_predictions\_df }\OtherTok{\textless{}{-}} \FunctionTok{pivot\_longer}\NormalTok{(stacked\_matrices\_df, }
                                    \SpecialCharTok{!}\FunctionTok{c}\NormalTok{(simulation\_num, poly\_degree), }
                                    \AttributeTok{values\_to =} \StringTok{"y"}\NormalTok{)}

\CommentTok{\# Now we can use ggplot2!}
\CommentTok{\# We just want to plot for degrees 1, 2, 10 and 20.}
\CommentTok{\# Add x{-}vector to dataframe:}
\NormalTok{plotting\_df }\OtherTok{\textless{}{-}} \FunctionTok{cbind}\NormalTok{(long\_predictions\_df, }\AttributeTok{x =}\NormalTok{ x) }\SpecialCharTok{\%\textgreater{}\%} 
  \FunctionTok{filter}\NormalTok{(poly\_degree }\SpecialCharTok{\%in\%} \FunctionTok{c}\NormalTok{(}\DecValTok{1}\NormalTok{, }\DecValTok{2}\NormalTok{, }\DecValTok{10}\NormalTok{, }\DecValTok{20}\NormalTok{)) }\CommentTok{\# Keep only the predictions for some degrees}

\FunctionTok{ggplot}\NormalTok{(plotting\_df, }\FunctionTok{aes}\NormalTok{(}\AttributeTok{x =}\NormalTok{ x, }\AttributeTok{y =}\NormalTok{ y, }\AttributeTok{group =}\NormalTok{ simulation\_num)) }\SpecialCharTok{+}
  \FunctionTok{geom\_line}\NormalTok{(}\FunctionTok{aes}\NormalTok{(}\AttributeTok{color =}\NormalTok{ simulation\_num)) }\SpecialCharTok{+}
  \FunctionTok{geom\_line}\NormalTok{(}\FunctionTok{aes}\NormalTok{(}\AttributeTok{x =}\NormalTok{ x, }\AttributeTok{y =}\NormalTok{ x}\SpecialCharTok{\^{}}\DecValTok{2}\NormalTok{), }\AttributeTok{size =} \FloatTok{1.5}\NormalTok{) }\SpecialCharTok{+}
  \FunctionTok{facet\_wrap}\NormalTok{(}\SpecialCharTok{\textasciitilde{}}\NormalTok{ poly\_degree) }\SpecialCharTok{+}
  \FunctionTok{theme\_bw}\NormalTok{() }\SpecialCharTok{+}
  \FunctionTok{theme}\NormalTok{(}\AttributeTok{legend.position =} \StringTok{"none"}\NormalTok{)}
\end{Highlighting}
\end{Shaded}

What do you observe in the produced plot? Which polynomial fits the best
to the true curve?

\begin{center}\rule{0.5\linewidth}{0.5pt}\end{center}

\subsubsection{b) Train and test MSE}\label{b-train-and-test-mse}

First we produce predictions at each grid point based on our training
data (\texttt{x} and \texttt{ymat}). Then we draw new observations to
calculate test MSE, see \texttt{testymat}.

Observe how \texttt{trainMSE} and \texttt{testMSE} are calculated, and
then run the code.

\begin{Shaded}
\begin{Highlighting}[]
\FunctionTok{set.seed}\NormalTok{(}\DecValTok{2}\NormalTok{) }\CommentTok{\# to reproduce}
\NormalTok{M }\OtherTok{\textless{}{-}} \DecValTok{100} \CommentTok{\# repeated samplings,x fixed but new errors}
\NormalTok{nord }\OtherTok{\textless{}{-}} \DecValTok{20}

\NormalTok{x }\OtherTok{\textless{}{-}} \FunctionTok{seq}\NormalTok{(}\AttributeTok{from =} \SpecialCharTok{{-}}\DecValTok{2}\NormalTok{, }\AttributeTok{to =} \DecValTok{4}\NormalTok{, }\AttributeTok{by =} \FloatTok{0.1}\NormalTok{)}
\NormalTok{truefunc }\OtherTok{\textless{}{-}} \ControlFlowTok{function}\NormalTok{(x) \{}
  \FunctionTok{return}\NormalTok{(x}\SpecialCharTok{\^{}}\DecValTok{2}\NormalTok{)}
\NormalTok{\}}
\NormalTok{true\_y }\OtherTok{\textless{}{-}} \FunctionTok{truefunc}\NormalTok{(x)}
\NormalTok{error }\OtherTok{\textless{}{-}} \FunctionTok{matrix}\NormalTok{(}\FunctionTok{rnorm}\NormalTok{(}\FunctionTok{length}\NormalTok{(x) }\SpecialCharTok{*}\NormalTok{ M, }\AttributeTok{mean =} \DecValTok{0}\NormalTok{, }\AttributeTok{sd =} \DecValTok{2}\NormalTok{), }\AttributeTok{nrow =}\NormalTok{ M, }\AttributeTok{byrow =} \ConstantTok{TRUE}\NormalTok{)}
\NormalTok{testerror }\OtherTok{\textless{}{-}} \FunctionTok{matrix}\NormalTok{(}\FunctionTok{rnorm}\NormalTok{(}\FunctionTok{length}\NormalTok{(x) }\SpecialCharTok{*}\NormalTok{ M, }\AttributeTok{mean =} \DecValTok{0}\NormalTok{, }\AttributeTok{sd =} \DecValTok{2}\NormalTok{), }\AttributeTok{nrow =}\NormalTok{ M, }\AttributeTok{byrow =} \ConstantTok{TRUE}\NormalTok{)}
\NormalTok{ymat }\OtherTok{\textless{}{-}} \FunctionTok{matrix}\NormalTok{(}\FunctionTok{rep}\NormalTok{(true\_y, M), }\AttributeTok{byrow =} \ConstantTok{TRUE}\NormalTok{, }\AttributeTok{nrow =}\NormalTok{ M) }\SpecialCharTok{+}\NormalTok{ error}
\NormalTok{testymat }\OtherTok{\textless{}{-}} \FunctionTok{matrix}\NormalTok{(}\FunctionTok{rep}\NormalTok{(true\_y, M), }\AttributeTok{byrow =} \ConstantTok{TRUE}\NormalTok{, }\AttributeTok{nrow =}\NormalTok{ M) }\SpecialCharTok{+}\NormalTok{ testerror}

\NormalTok{predictions\_list }\OtherTok{\textless{}{-}} \FunctionTok{lapply}\NormalTok{(}\DecValTok{1}\SpecialCharTok{:}\NormalTok{nord, matrix, }\AttributeTok{data =} \ConstantTok{NA}\NormalTok{, }\AttributeTok{nrow =}\NormalTok{ M, }\AttributeTok{ncol =} \FunctionTok{ncol}\NormalTok{(ymat))}
\ControlFlowTok{for}\NormalTok{ (i }\ControlFlowTok{in} \DecValTok{1}\SpecialCharTok{:}\NormalTok{nord) \{}
  \ControlFlowTok{for}\NormalTok{ (j }\ControlFlowTok{in} \DecValTok{1}\SpecialCharTok{:}\NormalTok{M) \{}
\NormalTok{    predictions\_list[[i]][j, ] }\OtherTok{\textless{}{-}} \FunctionTok{predict}\NormalTok{(}\FunctionTok{lm}\NormalTok{(ymat[j, ] }\SpecialCharTok{\textasciitilde{}} \FunctionTok{poly}\NormalTok{(x, i, }\AttributeTok{raw =} \ConstantTok{TRUE}\NormalTok{)))}
\NormalTok{  \}}
\NormalTok{\}}

\NormalTok{trainMSE }\OtherTok{\textless{}{-}} \FunctionTok{lapply}\NormalTok{(}\DecValTok{1}\SpecialCharTok{:}\NormalTok{nord, }
                   \ControlFlowTok{function}\NormalTok{(poly\_degree) \{}
                     \FunctionTok{rowMeans}\NormalTok{((predictions\_list[[poly\_degree]] }\SpecialCharTok{{-}}\NormalTok{ ymat)}\SpecialCharTok{\^{}}\DecValTok{2}\NormalTok{)}
\NormalTok{                   \}}
\NormalTok{)}
\NormalTok{testMSE }\OtherTok{\textless{}{-}} \FunctionTok{lapply}\NormalTok{(}\DecValTok{1}\SpecialCharTok{:}\NormalTok{nord, }
                  \ControlFlowTok{function}\NormalTok{(poly\_degree) \{}
                    \FunctionTok{rowMeans}\NormalTok{((predictions\_list[[poly\_degree]] }\SpecialCharTok{{-}}\NormalTok{ testymat)}\SpecialCharTok{\^{}}\DecValTok{2}\NormalTok{)}
\NormalTok{                  \}}
\NormalTok{)}
\end{Highlighting}
\end{Shaded}

Next, we plot the training and test error for each of the 100 data sets
we simulated, as well as two different plots that show the means across
the simulations.

\begin{Shaded}
\begin{Highlighting}[]
\FunctionTok{library}\NormalTok{(tidyverse) }\CommentTok{\# The tidyverse contains ggplot2, as well as tidyr and dplyr, }
\CommentTok{\# which we can use for dataframe manipulation.}

\CommentTok{\# Convert each matrix in the list form wide to long}
\CommentTok{\# (because that is the best format for ggplot2)}
\NormalTok{list\_train\_MSE }\OtherTok{\textless{}{-}} \FunctionTok{lapply}\NormalTok{(}\DecValTok{1}\SpecialCharTok{:}\NormalTok{nord, }\ControlFlowTok{function}\NormalTok{(poly\_degree) \{}
  \FunctionTok{cbind}\NormalTok{(}\AttributeTok{error =}\NormalTok{ trainMSE[[poly\_degree]], }
\NormalTok{        poly\_degree, }
        \AttributeTok{error\_type =} \StringTok{"train"}\NormalTok{,}
        \AttributeTok{simulation\_num =} \DecValTok{1}\SpecialCharTok{:}\NormalTok{M)}
\NormalTok{\})}
\NormalTok{list\_test\_MSE }\OtherTok{\textless{}{-}} \FunctionTok{lapply}\NormalTok{(}\DecValTok{1}\SpecialCharTok{:}\NormalTok{nord, }\ControlFlowTok{function}\NormalTok{(poly\_degree) \{}
  \FunctionTok{cbind}\NormalTok{(}\AttributeTok{error =}\NormalTok{ testMSE[[poly\_degree]], }
\NormalTok{        poly\_degree, }
        \AttributeTok{error\_type =} \StringTok{"test"}\NormalTok{, }
        \AttributeTok{simulation\_num =} \DecValTok{1}\SpecialCharTok{:}\NormalTok{M)}
\NormalTok{\})}

\CommentTok{\# Now predictions\_list is a list with 20 entries, where each entry is a matrix }
\CommentTok{\# with 100 rows, where each row is the predicted polynomial of that degree.}

\NormalTok{stacked\_train }\OtherTok{\textless{}{-}} \ConstantTok{NULL}
\ControlFlowTok{for}\NormalTok{ (i }\ControlFlowTok{in} \DecValTok{1}\SpecialCharTok{:}\NormalTok{nord) \{}
\NormalTok{  stacked\_train }\OtherTok{\textless{}{-}}
    \FunctionTok{rbind}\NormalTok{(stacked\_train, list\_train\_MSE[[i]])}
\NormalTok{\}}
\NormalTok{stacked\_test }\OtherTok{\textless{}{-}} \ConstantTok{NULL}
\ControlFlowTok{for}\NormalTok{ (i }\ControlFlowTok{in} \DecValTok{1}\SpecialCharTok{:}\NormalTok{nord) \{}
\NormalTok{  stacked\_test }\OtherTok{\textless{}{-}}
    \FunctionTok{rbind}\NormalTok{(stacked\_test, list\_test\_MSE[[i]])}
\NormalTok{\}}

\NormalTok{stacked\_errors\_df }\OtherTok{\textless{}{-}} \FunctionTok{as.data.frame}\NormalTok{(}\FunctionTok{rbind}\NormalTok{(stacked\_train, stacked\_test))}
\CommentTok{\# This is already on long format. }
\NormalTok{stacked\_errors\_df}\SpecialCharTok{$}\NormalTok{error }\OtherTok{\textless{}{-}} \FunctionTok{as.numeric}\NormalTok{(stacked\_errors\_df}\SpecialCharTok{$}\NormalTok{error)}
\NormalTok{stacked\_errors\_df}\SpecialCharTok{$}\NormalTok{simulation\_num }\OtherTok{\textless{}{-}} \FunctionTok{as.integer}\NormalTok{(stacked\_errors\_df}\SpecialCharTok{$}\NormalTok{simulation\_num)}
\NormalTok{stacked\_errors\_df}\SpecialCharTok{$}\NormalTok{poly\_degree }\OtherTok{\textless{}{-}} \FunctionTok{as.integer}\NormalTok{(stacked\_errors\_df}\SpecialCharTok{$}\NormalTok{poly\_degree)}

\NormalTok{p.all\_lines }\OtherTok{\textless{}{-}} \FunctionTok{ggplot}\NormalTok{(}\AttributeTok{data =}\NormalTok{ stacked\_errors\_df, }
                      \FunctionTok{aes}\NormalTok{(}\AttributeTok{x =}\NormalTok{ poly\_degree, }\AttributeTok{y =}\NormalTok{ error, }\AttributeTok{group =}\NormalTok{ simulation\_num)) }\SpecialCharTok{+}
  \FunctionTok{geom\_line}\NormalTok{(}\FunctionTok{aes}\NormalTok{(}\AttributeTok{color =}\NormalTok{ simulation\_num)) }\SpecialCharTok{+}
  \FunctionTok{facet\_wrap}\NormalTok{(}\SpecialCharTok{\textasciitilde{}}\NormalTok{ error\_type) }\SpecialCharTok{+}
  \FunctionTok{xlab}\NormalTok{(}\StringTok{"Polynomial degree"}\NormalTok{) }\SpecialCharTok{+}
  \FunctionTok{ylab}\NormalTok{(}\StringTok{"MSE"}\NormalTok{) }\SpecialCharTok{+}
  \FunctionTok{theme\_bw}\NormalTok{() }\SpecialCharTok{+}
  \FunctionTok{theme}\NormalTok{(}\AttributeTok{legend.position =} \StringTok{"none"}\NormalTok{)}

\NormalTok{p.bars }\OtherTok{\textless{}{-}} \FunctionTok{ggplot}\NormalTok{(stacked\_errors\_df, }\FunctionTok{aes}\NormalTok{(}\AttributeTok{x =} \FunctionTok{as.factor}\NormalTok{(poly\_degree), }\AttributeTok{y =}\NormalTok{ error)) }\SpecialCharTok{+}
  \FunctionTok{geom\_boxplot}\NormalTok{(}\FunctionTok{aes}\NormalTok{(}\AttributeTok{fill =}\NormalTok{ error\_type)) }\SpecialCharTok{+}
  \FunctionTok{scale\_fill\_discrete}\NormalTok{(}\AttributeTok{name =} \StringTok{"Error type"}\NormalTok{) }\SpecialCharTok{+}
  \FunctionTok{xlab}\NormalTok{(}\StringTok{"Polynomial degree"}\NormalTok{) }\SpecialCharTok{+}
  \FunctionTok{ylab}\NormalTok{(}\StringTok{"MSE"}\NormalTok{) }\SpecialCharTok{+}
  \FunctionTok{theme\_bw}\NormalTok{()}

\CommentTok{\# Here we find the average test error and training error across the repeated simulations. }
\CommentTok{\# The symbol "\%\textgreater{}\%" is called a pipe, and comes from the tidyverse packages, }
\CommentTok{\# which provide convenient functions for working with data frames.}
\NormalTok{means\_across\_simulations }\OtherTok{\textless{}{-}}\NormalTok{ stacked\_errors\_df }\SpecialCharTok{\%\textgreater{}\%} 
  \FunctionTok{group\_by}\NormalTok{(error\_type, poly\_degree) }\SpecialCharTok{\%\textgreater{}\%} 
  \FunctionTok{summarise}\NormalTok{(}\AttributeTok{mean =} \FunctionTok{mean}\NormalTok{(error))}

\NormalTok{p.means }\OtherTok{\textless{}{-}} \FunctionTok{ggplot}\NormalTok{(means\_across\_simulations, }\FunctionTok{aes}\NormalTok{(}\AttributeTok{x =}\NormalTok{ poly\_degree, }\AttributeTok{y =}\NormalTok{ mean)) }\SpecialCharTok{+}
  \FunctionTok{geom\_line}\NormalTok{(}\FunctionTok{aes}\NormalTok{(}\AttributeTok{color =}\NormalTok{ error\_type)) }\SpecialCharTok{+}
  \FunctionTok{scale\_color\_discrete}\NormalTok{(}\AttributeTok{name =} \StringTok{"Error type"}\NormalTok{) }\SpecialCharTok{+}
  \FunctionTok{xlab}\NormalTok{(}\StringTok{"Polynomial degree"}\NormalTok{) }\SpecialCharTok{+}
  \FunctionTok{ylab}\NormalTok{(}\StringTok{"MSE"}\NormalTok{) }\SpecialCharTok{+}
  \FunctionTok{theme\_bw}\NormalTok{()}

\FunctionTok{library}\NormalTok{(patchwork) }\CommentTok{\# The library patchwork is the best way of combining ggplot2 objects. }
\CommentTok{\# You could also use the function ggarrange from the ggpubr package.}

\NormalTok{p.all\_lines }\SpecialCharTok{/}\NormalTok{ (p.bars }\SpecialCharTok{+}\NormalTok{ p.means)}
\end{Highlighting}
\end{Shaded}

\begin{itemize}
\tightlist
\item
  Which polynomial degree gives the smallest mean testMSE?
\item
  Which polynomial degree gives the smallest mean trainMSE?
\item
  Which should you use to predict a new value of \(y\)?
\end{itemize}

\subsubsection{c) Bias and variance - we use the
truth!}\label{c-bias-and-variance---we-use-the-truth}

Finally, we want to see how the expected quadratic loss can be
decomposed into

\begin{itemize}
\tightlist
\item
  irreducible error: \(\text{Var}(\varepsilon)=4\)
\item
  squared bias: difference between mean of estimated parametric model
  chosen and the true underlying curve (\texttt{truefunc})
\item
  variance: variance of the estimated parametric model
\end{itemize}

Notice that the test data is not used -- only predicted values in each x
grid point.

Study and run the code. Explain the plots produced.

\begin{Shaded}
\begin{Highlighting}[]
\NormalTok{meanmat }\OtherTok{\textless{}{-}} \FunctionTok{matrix}\NormalTok{(}\AttributeTok{ncol =} \FunctionTok{length}\NormalTok{(x), }\AttributeTok{nrow =}\NormalTok{ nord)}
\NormalTok{varmat }\OtherTok{\textless{}{-}} \FunctionTok{matrix}\NormalTok{(}\AttributeTok{ncol =} \FunctionTok{length}\NormalTok{(x), }\AttributeTok{nrow =}\NormalTok{ nord)}
\ControlFlowTok{for}\NormalTok{ (j }\ControlFlowTok{in} \DecValTok{1}\SpecialCharTok{:}\NormalTok{nord) \{}
  \CommentTok{\# We now take the mean over the M simulations {-}}
  \CommentTok{\# to mimic E and Var at each x value and each poly model}
\NormalTok{  meanmat[j, ] }\OtherTok{\textless{}{-}} \FunctionTok{apply}\NormalTok{(predictions\_list[[j]], }\DecValTok{2}\NormalTok{, mean)}
\NormalTok{  varmat[j, ] }\OtherTok{\textless{}{-}} \FunctionTok{apply}\NormalTok{(predictions\_list[[j]], }\DecValTok{2}\NormalTok{, var)}
\NormalTok{\}}

\CommentTok{\# Here the truth is finally used!}
\NormalTok{bias2mat }\OtherTok{\textless{}{-}}\NormalTok{ (meanmat }\SpecialCharTok{{-}} \FunctionTok{matrix}\NormalTok{(}\FunctionTok{rep}\NormalTok{(true\_y, nord), }\AttributeTok{byrow =} \ConstantTok{TRUE}\NormalTok{, }\AttributeTok{nrow =}\NormalTok{ nord))}\SpecialCharTok{\^{}}\DecValTok{2} 
\end{Highlighting}
\end{Shaded}

Plotting the polys as a function of x:

\begin{Shaded}
\begin{Highlighting}[]
\NormalTok{df }\OtherTok{\textless{}{-}} \FunctionTok{data.frame}\NormalTok{(}\AttributeTok{x =} \FunctionTok{rep}\NormalTok{(x, }\AttributeTok{each =}\NormalTok{ nord), }\AttributeTok{poly\_degree =} \FunctionTok{rep}\NormalTok{(}\DecValTok{1}\SpecialCharTok{:}\NormalTok{nord, }\FunctionTok{length}\NormalTok{(x)), }
                 \AttributeTok{bias2 =} \FunctionTok{c}\NormalTok{(bias2mat), }\AttributeTok{variance =} \FunctionTok{c}\NormalTok{(varmat), }
                 \AttributeTok{irreducible\_error =} \FunctionTok{rep}\NormalTok{(}\DecValTok{4}\NormalTok{, }\FunctionTok{prod}\NormalTok{(}\FunctionTok{dim}\NormalTok{(varmat)))) }\CommentTok{\#irr is just 1}

\NormalTok{df}\SpecialCharTok{$}\NormalTok{total }\OtherTok{\textless{}{-}}\NormalTok{ df}\SpecialCharTok{$}\NormalTok{bias2 }\SpecialCharTok{+}\NormalTok{ df}\SpecialCharTok{$}\NormalTok{variance }\SpecialCharTok{+}\NormalTok{ df}\SpecialCharTok{$}\NormalTok{irreducible\_error}

\NormalTok{df\_long }\OtherTok{\textless{}{-}} \FunctionTok{pivot\_longer}\NormalTok{(df, }\AttributeTok{cols =} \SpecialCharTok{!}\FunctionTok{c}\NormalTok{(x, poly\_degree), }\AttributeTok{names\_to =} \StringTok{"type"}\NormalTok{) }

\NormalTok{df\_select\_poly }\OtherTok{\textless{}{-}} \FunctionTok{filter}\NormalTok{(df\_long, poly\_degree }\SpecialCharTok{\%in\%} \FunctionTok{c}\NormalTok{(}\DecValTok{1}\NormalTok{, }\DecValTok{2}\NormalTok{, }\DecValTok{10}\NormalTok{, }\DecValTok{20}\NormalTok{))}

\FunctionTok{ggplot}\NormalTok{(df\_select\_poly, }\FunctionTok{aes}\NormalTok{(}\AttributeTok{x =}\NormalTok{ x, }\AttributeTok{y =}\NormalTok{ value, }\AttributeTok{group =}\NormalTok{ type)) }\SpecialCharTok{+}
  \FunctionTok{geom\_line}\NormalTok{(}\FunctionTok{aes}\NormalTok{(}\AttributeTok{color =}\NormalTok{ type)) }\SpecialCharTok{+}
  \FunctionTok{facet\_wrap}\NormalTok{(}\SpecialCharTok{\textasciitilde{}}\NormalTok{poly\_degree, }\AttributeTok{scales =} \StringTok{"free"}\NormalTok{, }\AttributeTok{labeller =}\NormalTok{ label\_both) }\SpecialCharTok{+}
  \FunctionTok{theme\_bw}\NormalTok{()}
\end{Highlighting}
\end{Shaded}

Now plotting effect of more complex model at 4 chosen values of x,
compare to Figures in 2.12 on page 36 in ISL (our textbook).

\begin{Shaded}
\begin{Highlighting}[]
\NormalTok{df\_select\_x }\OtherTok{\textless{}{-}} \FunctionTok{filter}\NormalTok{(df\_long, x }\SpecialCharTok{\%in\%} \FunctionTok{c}\NormalTok{(}\SpecialCharTok{{-}}\DecValTok{1}\NormalTok{, }\FloatTok{0.5}\NormalTok{, }\DecValTok{2}\NormalTok{, }\FloatTok{3.5}\NormalTok{))}

\FunctionTok{ggplot}\NormalTok{(df\_select\_x, }\FunctionTok{aes}\NormalTok{(}\AttributeTok{x =}\NormalTok{ poly\_degree, }\AttributeTok{y =}\NormalTok{ value, }\AttributeTok{group =}\NormalTok{ type)) }\SpecialCharTok{+}
  \FunctionTok{geom\_line}\NormalTok{(}\FunctionTok{aes}\NormalTok{(}\AttributeTok{color =}\NormalTok{ type)) }\SpecialCharTok{+}
  \FunctionTok{facet\_wrap}\NormalTok{(}\SpecialCharTok{\textasciitilde{}}\NormalTok{x, }\AttributeTok{scales =} \StringTok{"free"}\NormalTok{, }\AttributeTok{labeller =}\NormalTok{ label\_both) }\SpecialCharTok{+}
  \FunctionTok{theme\_bw}\NormalTok{()}
\end{Highlighting}
\end{Shaded}

Study the final plot you produced: when the flexibility increases (poly
increase), what happens with i) the squared bias, ii) the variance, iii)
the irreducible error?

\begin{center}\rule{0.5\linewidth}{0.5pt}\end{center}

\subsubsection{d) Repeat a-c}\label{d-repeat-a-c}

Try to change the true function \texttt{truefunc} to something else -
maybe order \(3\)? What does this do the the plots produced? Maybe you
then also want to plot poly3?

Also try to change the standard deviation of the noise added to the
curve (now it is sd=2). What happens if you change this to \texttt{sd=1}
or \texttt{sd=3}?

Or, change to the true function so that is not a polynomial?

\begin{center}\rule{0.5\linewidth}{0.5pt}\end{center}

\section{Acknowledgements}\label{acknowledgements}

We thank Mette Langaas and her PhD students (in particular Julia Debik)
from 2018 and 2019 for building up the original version of this exercise
sheet.

\end{document}
